\section{Conclusão}

\begin{frame}{Considerações finais}
Neste trabalho, foi proposta a seguinte questão de pesquisa: 
\linebreak
\linebreak
\begin{center}
"É possível desenvolver um sistema que simule o comportamento do som dentro de um ambiente fechado utilizando uma abordagem multiagentes?" 
\linebreak
\linebreak
\end{center}
Com base nos \textbf{resultados} desse trabalho, oriundos dos \textbf{objetivos específicos}, pode-se concluir que a resposta à questão é: \textbf{Sim}, o simulador atendeu ao objetivo geral para o qual foi proposto.
\end{frame}

\begin{frame}[allowframebreaks]{Trabalhos futuros}
\begin{enumerate}
\item Alternativa ao JFreeChart que possibilite a representação de todos os elementos dentro do ambiente.
\linebreak
\item Incorporar catálogo de materiais associados aos seus respectivos índices de absorção.
\linebreak
\item Adicionar novos parâmetros a serem coletados durante a simulação e apresentados ao usuário final, como por exemplo: clareza, brilho, equilíbrio, ruído e distorção \cite{figueiredo}.
\linebreak
\item Refinar cálculos de reflexão sonora.
\linebreak
\linebreak
\end{enumerate}

A arquitetura do simulador foi elaborada de modo a garantir uma ferramenta manutenível e extensível. O código fonte está disponível para a comunidade de software livre sob a licença GPL v2 no GitHub em: \url{https://github.com/gabriel-augusto/AcousticSimulator}.

\end{frame}

\begin{frame}
\begin{center}
\textbf{{\Huge Obrigado.}}
\end{center}
\end{frame}