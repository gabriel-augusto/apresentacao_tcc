\section{Introdução}

% ------------------------------ FRAME 1 ----------------------------- %
\begin{frame}{Contextualização}
  \begin{itemize}
  	\item Estúdios musicais
  	\linebreak
  	\item Auditórios
	\linebreak
  	\item Salas de aula
  \end{itemize}
  
\end{frame}
% ------------------------------ FRAME 2 ----------------------------- %
\begin{frame}{Questão de pesquisa}

	\begin{center}
		\textit{"É possível desenvolver um sistema que simule o comportamento do som dentro de um ambiente fechado utilizando uma abordagem multiagentes?"}
	\end{center}

\end{frame}
% ------------------------------ FRAME 3 ----------------------------- %
\begin{frame}{Objetivo geral}

\textbf{Desenvolver} um sistema que seja capaz de \textbf{simular o comportamento do som} dentro de um ambiente fechado, utilizando uma abordagem \textbf{multiagentes}, para que possa ser \textbf{incorporado} na implementação de \textbf{novos simuladores acústicos}, afim de potencializar o auxílio aos projetistas e/ou especialistas em acústica no que tange o acompanhamento e a avaliação dos parâmetros de seus projetos.

\end{frame}
% ------------------------------ FRAME 4 ----------------------------- %
\begin{frame}[allowframebreaks]{Objetivos específicos}

\begin{enumerate}
\item Estudar o \textbf{comportamento do som} dentro de ambientes acústicos, identificando \textbf{variáveis acústicas} presentes dentro desses ambientes.

\bigskip 

\item Identificar \textbf{índices de absorção} referentes aos \textbf{materiais} presentes nos ambientes em análise.

\bigskip 

\item Propor um \textbf{suporte tecnológico} baseado em uma abordagem \textbf{multiagentes}, o qual será utilizado para a implementação da solução.

\bigskip 

\item Explorar \textbf{técnicas de programação, padrões de projeto} e demais \textbf{boas práticas} da Engenharia de Software visando o desenvolvimento de um simulador \textbf{manutenível e extensível}.

\bigskip 

\item Definir \textbf{métricas de qualidade} visando realizar a análise estática e a cobertura do código do simulador proposto, com base em uma abordagem de teste apropriada para o contexto, focada, principalmente, \textbf{em testes unitários}.
\end{enumerate}

\end{frame}